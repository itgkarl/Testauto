\chapter{Grundlagen}
\label{sec:grundlagen}


\section{Softwarequalität}
\label{sec:softwarequalität}

Nach der Norm ISO-25000:2014 4.33 bezeichnet der Begriff der Softwarequalität die Fähigkeit einer Software die expliziten und impliziten Bedürftnisse von Benutzern, unter den Bedingungen unter sie benutzt wird, zu befriedigen.  \cite {international_organization_for_standardization_iso_so/iec_2014}
Softwarequalität hat nach dieser Definition einen subjektiven Charakter.
Dieser subjektive Charakter macht den Begriff nicht operabel und daher in der Praxis nicht direkt anwendbar. \cite[S. 257]{balzert_lehrbuch_1998} Aus diesem Grund existieren sogenannte Qualitätsmodelle, die den Begriff der Softwarequalität messbar und damit auch überprüfbar machen sollen.
Ein solches Qualitätsmodell wird zum Beispiel in der ISO-Norm 9126 vorgestellt.
Hier werden verschiedene Qualitätsmerkmale deffiniert die zur Beurteilung der Gesamtqualität eines Softwareprodukts dienen.
Hierunter fallen die Merkmale:

\begin{itemize}
	  \itemsep0pt
      \item Funktionalität
      \item Zuverlässigkeit
      \item Benutzbarkeit
      \item Effizienz
      \item Änderbarkeit
      \item Übertragbarkeit          
\end{itemize}

Es existieren verschiedene Methoden um sicherzustellen, dass Software bezogen auf die Qualitätsmerkmale gewissen Anforderungen genügt.
Eine Gruppe geht dabei davon aus, dass ein qualitativ hochwertiger Prozess der Produkterstellung die Entstehung von qualitativ hochwertigen Produkten begünstigt.
Das Augenmerk wird hierbei also auf die Prozessqualität gelegt.
Allgemein fasst man diese Gruppe unter dem Begriff des prozessorientiertes Qualitätsmanagement zusammen. Hierunter fallen unter anderem die klassischen Vorgehensmodelle der Softwarentwicklung.
Worauf sich diese Arbeit jedoch konzentrieren möchte sind die Methoden des produktorientierten Qualitätsmanagement.
Hierbei wird das Softwareprodukt direkt bezüglich der Qualitätsmerkmale überprüft.
Darunter fallen beispielsweise Softwaretests. Mittels Softwaretests das konkrete Produkt direkt untersucht damit es die gestellten Qualitätsanforderungen möglichst gut erfüllt.
 