\chapter{Testautomatisierung}
\label{sec:testautomatisierung}

Nach Seidl et al. versteht man unter dem Begriff der Testautomatisierung \glqq die Durchführung von ansonsten manuellen Testtätigkeiten durch Automaten.\grqq\ \cite[Seite 7]{seidl_basiswissen_2012}
Das Spektrum der Testautomatisierung umfasst demnach alle Tätigkeiten die dazu dienen die Qualität einer Software zu Überprüfen. Darunter fallen alle Aufgaben die im Testprozess in den einzelnen Phasen des Softwareentwicklung anfallen.
Testautomatisierung ist also nicht nur auf die automatisierte Testdurchführung beschränkt, sondern kann ebenso bei der Testfallerstellung, der Testdatengenerierung, der Testauswertung oder auch der Testumgebungsherstellung und -wiederherstellung eine Rolle spielen.
\glqq Die Grenze der Automatisierung liegen [nach Seidl et al.] darin, dass diese nur die manuellen Tätigkeiten eines Testers übernehmen kann, nicht aber die intellektuelle, kreative und intuitive Dimension dieser Rolle.\grqq\ \cite[Seite 7]{seidl_basiswissen_2012}
Besonders lohnenswert ist eine Automatisierung bei sich wiederholenden Aufgaben, zu denen
besonders die Testdurchführung zählt. Die Automatisierung in diesem Bereich ist bereits weit verbreitet und soll auch den Fokus dieser Arbeit bilden. Die anderen Bereiche der Testautomatisierung sollen nur kurz aufgezeigt werden.

\section{Warum Testautomatisierung?}
\label{sec:warum_testautomatisierung}

Studien haben gezeigt, dass das Testen für 50\% und mehr der gesamten Projektkosten verantwortlich ist. \cite{ramler_economic_2006} 
Es gab daher immer wieder Versuch den Bereich des Testens zu optimieren. Testautomatisierung ist ein weit verbreiteter Ansatz die Kosten von manuellen Tests zu reduzieren und dabei die Qualität des Softwaresystems zu sichern. \cite{amannejad_search-based_2014} Man verspricht sich dabei eine Reihe von Vorteilen gegenüber der manuellen Ausführung von Testfällen. Fewster und Graham haben dazu eine Liste aufgestellt die in Tabelle \ref{tbl:vorteile_testautomatisierung} verkürzt dargestellt sind.
Ähnliche Vorteile werden auch von Thalle \cite[Seite 228]{thaller_software-test_2002} beschrieben.

\begin{table}
\begin{tabular}{p{0.4\textwidth}|p{0.6\textwidth}}
Führe existierende Regressionstests für eine neue Version eines Programms aus.
& Die Möglichkeit bereits erstellte Testfälle ohne Mehraufwand auszuführen macht das Testen effektiver. \\
\hline 
Führe mehr Tests öfter aus .
& Automatisierung bedeutet schnellere Testausführung. Dadurch lassen sich mehr Testdurchläufe bewerkstelligen. Automatisierung sollte auch das erstellen neuer Testfälle einfacher und schneller machen. \\ 
\hline 
Führe Tests durch die manuell schwer bis unmöglich wären. & 
Performancetests sind beispielsweise ohne Automatisierung fast nicht zu bewältigen.\\ 
\hline 
Bessere verwendung von Resourcen. & Die Automatisierung von sich wiederholenden Aufgaben ermöglicht die Testern die Arbeit an anderen Aufgaben. \\ 
\hline 
Wiederholbarkeit und Konsistenz von Testfällen. & Tests werden immer gleich ausgeführt. Auf diese weise können die Testergebnisse besser verglichen werden. \\ 
\hline 
Wiederverwendbarkeit von Tests. & Vor allem neuen Projekte werden durch die Wiederverwendbarkeit von Testfällen beschleunigt. \\ 
\hline 
Frühere Markteinführung. & Das Wiederverwenden und beschleunigen von Testfällen beschläunigt den gesamten Testprozess. Das verkürzt letztendlich auch die Zeit bis zur Markteinführung. \\ 
\hline 
Verbessertes Vertrauen. & Viele Testfälle die oft und konstant ausgeführt werden können erhöhen das Vertrauen in die Software und seine Marktreife.\\ 

\end{tabular} 
\caption{Vorteile Testautomatisierung nach Fewster und Graham \cite{fewster_software_1999}}
\label{tbl:vorteile_testautomatisierung}
\end{table}



Die meisten der Aufgelisteten Vorteile können mit den Worten Effizienz und Wiederverwertbarkeit zusammengefasst werden.
Die Testautomatisierung entfaltet ihr volles Potential immer dann wenn Testfälle nicht nur einmal, sondern wiederholt ausgeführt werden. Mittels Automatisierung können Testfälle wiederholt durchlaufen werden ohne dabei einen großen Mehraufwand zu erzeugen. Tester können so entlastet werden und sich anderen Aufgaben widmen. Regressionstests eignen sich daher beispielsweise besonders gut für eine Automatisierung. Andere Bereiche wie zum Beispiel Stresstests wären ohne Automatisierung schwer bis gar nicht zu realisieren. Ein Stresstest mit ca. 200 gleichzeitigen Benutzern ist auf manuelle Weise nicht umsetzbar. Die eingaben von 200 Benutzern können hingegen recht einfach mittels automatisierten Tests simuliert werden.
Mit Hilfe der Testautomatisierung ist es daher möglich die Zeit die das Testen benötigt zu verringern und dabei die Softwarequalität zu erhöhen. \newline
All die genannten Vorteile lassen die Testautomatisierung sehr attraktiv erscheinen. Diese Versprechen sind in der Praxis jedoch nicht leicht zu erreichen. Wird die Automatisierung nicht gut umgesetzt kann die Testautomatisierung schnell zu einer größeren Belastung werden als dass sie Nutzen bringt.
Fewster und Graham haben dazu auch hierzu eine Liste mit bekannten Problemen zusammengestellt. \cite{fewster_software_1999} die in Tabelle \ref{tbl:nachteile_testautomatisierung} zusammengefasst wurden.
Das Hauptproblem ist dabei meist, dass der Aufwand der Testautomatisierung unterschätzt wird. In jedem etwas größeren Projekt muss die Testautomatisierung also eigenes Projekt gesehen werden. Jedes Projekt erfordert eine genaue Planung und muss gewissen Prozessen folgen. Wird diese Planung vernachlässigt oder die Prozesse missachtet kann ein Testautomatisierungsprojekt schnell in die falsche Richtung laufen.
Vor allem die Planung ist hier besonders wichtig. Nicht alles was Automatisiert werden kann sollte auch automatisiert werden.
Amannejad et al. \cite{amannejad_search-based_2014} widmen sich in einem Paper der Frage, welche Teile eines Testobjekts in eine automatisierten Art und Weise getestet werden sollten. Und kommen zu eben diesem Ergebnis. Die Vorteile durch die Testautomatisierung überwiegen nicht immer und sind gegen die zu erwartenden Aufwende zu prüfen. Erst wenn die zu erwartenden Einsparungen die Kosten überwiegen ist die Testautomatisierung sinnvoll.



\begin{table}
\begin{tabular}{p{0.4\textwidth}|p{0.6\textwidth}}
Unrealistische Erwartungen.
& Manager erwarten oft, dass die Testautomatiesierung alle Probleme löst und sofort die Qualität der Software verbessert wird. \\
\hline 
Schlechte Testpraxis.
& Wenn die Testpraktiken bereits schlecht sind ist es besser diese zunächst zu verbessern bevor mit einer Automatisierung begonnen wird. \\ 
\hline 
Erwartung, dass automatisierte Tests viele neue Fehler findet. & 
Wenn automatisierte tests einmal erfolgreich ausgeführt wurden, ist es nicht sehr warscheinlich, dass sie in folgenden Testläufen noch viele weitere Fehler finden werden.\\ 
\hline 
Falsche Vorstellung von Sicherheit. & Ein Testreport ohne Fehler bedeutet nicht, dass das Testobjekt keine Fehler hat.  \\ 
\hline 
Wartung. & Wenn das Testobjekt geändert wird, müssen meist auch die automatisierten Teställe angepasst werden. Wenn die Wartung mehr zeit Verschlingt als durch die Automatisierung eingespart werden kann ist der Nutzen fraglich.  \\ 
\hline 
Technische Probleme. & Testfälle zu automatisieren ist keine einfache Aufgabe. Es ist zu erwarten, dass dabei eine reihe von Problemen gelöst werden müssen. \\ 
\hline 
Organisatorische fragen. & Eine Erfolgreiche Testautomatisierung erfordert einen hohen Grad an technischer Kompetenz und Unterstützung des Managments. Testautomatisierung hat drüber hinaus Einfluss auf die Organisation und erfordert oft Änderungen in den etablierten Prozessen. \\ 
\end{tabular} 
\caption{Nachteile Testautomatisierung nach Fewster und Graham \cite{fewster_software_1999}}
\label{tbl:nachteile_testautomatisierung}
\end{table}





\section{Bereicher der Testautomatisierung}
\label{sec:bereicher_der_estautomatisierung}

Die Vor- und Nachteile der Testautomatisierung wie sie in Kapitel \ref{sec:warum_testautomatisierung} beschreiben sind beziehen sich Hauptsächlich auf die automatisierte Durchführung von Testfällen. Hierauf soll auch der Fokus dieser Arbeit liegen. Wie aber bereits eingangs in Kapitel \ref{sec:testautomatisierung} erwähnt, erstreckt sich die Möglichkeit zur Testautomatisierung über den gesamten Bereich des Testprozesses.
Eine dem Testprozess recht ähnliche jedoch leicht andere Unterteilung für die Bereiche der Testautomatisierung schlägt Amannejad et al. \cite{amannejad_search-based_2014} vor.
Er Unterteilt die Testautomatisierung im Testprozess in vier Aufgaben:

\begin{itemize}
	  \itemsep0pt
      \item Testdesign: Erstellen einer Liste von Testfällen um gewisse Akzeptanzkriterien zu prüfen.
      \item Testerstellung: Erstellen von automatisiertem Testcode.
      \item Testausführung: Ausführen von Testfällen und aufzeichnen der Ergebnisse.
      \item Testauswertung: Auswerten der aufgezeichneten Testergebnisse.         
\end{itemize}

\subsection{Testdesign}
\label{subsec:testdesign}
Unter Testdesign versteht man das erstellen einer Liste von Testfällen um gewisse Akzeptanzkriterien zu prüfen. Darunter fällt auch das identifizieren von Testdaten wie mögliche Eingabewerte und erwartete Ergebnisse. Hierfür gibt es zahlreiche Ansätze die manuell, aber auch toolgestützt und damit automatisiert durchgeführt werden können.
Unter dem Oberbegriff er Kombinatorik existieren einige Testfallentwurfsmethoden die vor allem darauf abzielen eine sinnvolle Auswahl ein Eingabedaten zu ermitteln.
Nach Seid et al. fallen darunter: \cite[Seite 27]{seidl_basiswissen_2012}
\begin{itemize}
	  \itemsep0pt
      \item Äquivalenzklassenbildung.
      \item Grenzwertanalyse.
      \item Klassifikationsbaummethode.
\end{itemize}
Fewster und Graham \cite[Seite 19 ff.]{fewster_software_1999} beschreiben eine weitere Möglichkeit bei der Tools direkt den Code und die Interfaces des zu testenden Systems verwenden um Eingabedaten zu Generieren.
Ein weiterer weg zum automatisierten Design von Testfällen den Seidel et al. aufzeigen \cite[Seite 33]{seidl_basiswissen_2012} sind Modellbasierte Techniken. In Modellbasierten Techniken wird das zu testende System in einem hohen Detailgrad durch Modelle abgebildet. Mit Hilfe dieser Modelle ist es möglich Testfälle abzuleiten. 

data mining approach \cite{last_data_2003}
Goal-driven Approach \cite{memon_using_1999}
requirement-based test generatio \cite{tahat_requirement-based_2001}

\subsection{Testerstellung}
\label{subsec:testerstellung}
Capture and reply --> verwenden für datadriven. Keyworddriven
Über Modellbasierte Techniken ist es nicht nur möglich Testfalldesigns abzuleiten. Bei entsprechend hohen Detailgrad der Modelle ist auch eine automatisierte Erzeugung von Testfällen möglich.
Bouquet et al. \cite{bouquet_test_2008} beschreiben beispielsweise eine Modellbasiertes Framework welches das automatisierte designen, generieren, managen und ausführen von Testfällen unterstützt.
Oft werden dazu unterstützend sogenannte Testorakel benutzt. Unter einem Testorakel versteht man eine Quelle die Auskunft über einen zu erwartendes Ergebnis in einem Testfall gibt. Es gibt zahlreiche Ansätze diese Testorakel in unterschiedlichen Bereichen wie Spezifikation und GUI zu automatisieren. \cite{memon_automated_2000} \cite{richardson_specification-based_1992}
\cite{shahamiri_comparative_2009} Diese Automatisierten Testorakel können bei der automatisierten Testfallerzeugung als Quelle für die Erwarteten Ergebnisse dienen.
Für die Erzeugung der eigentlichen Testfälle können wie schon beim Testfalldesign modellbasierte Ansätze verwendet werden.
Auch Daten getriebene Ansätze sind möglich.
Keworddriven

\subsection{Testausführung}
\label{subsec:testausführung}


\subsection{Testauswertung}
\label{subsec:testauswertung}
By consulting several books a
nd online resources [17-19]
on software testing and incorpo
rating different views and
classifications, we divide the tes
ting tasks into four types:
1. Test-case design: Designating the list of test cases or
test requirements to satisfy coverage criteria, or other
engineering goals.
2. Test scripting: Documenting test cases in manual test
scripts or automated test code.
3. Test execution: Running test cases on the software
under test and recording the outputs.
4. Test evaluation: Evaluat
ing results of testing (pass or
fail), also known as test oracle or test verdict (Searchbased III)


\section{Selenium}
\label{sec:selenium}

B
ASICS OF
S
ELENIUM AND
J
METER
Selenium Remote Control (RC) is allows you to write
automated web application UI tests in any programming
language against any HTTP website using any mainstream
JavaScript-enabled browser [7].
The Selenium Server which launches and kills browsers,
interprets and runs the Selenese commands passed from the
test program, and acts as an HTTP proxy, intercepting and
verifying HTTP messages passed between the browser and
the AUT. (A Test automatisation Framework)