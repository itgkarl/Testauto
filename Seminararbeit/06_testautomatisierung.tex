\chapter{Testautomatisierung}
\label{sec:testautomatisierung}

Nach Seidl et al. versteht man unter dem Begriff der Testautomatisierung \glqq die Durchführung von ansonsten manuellen Testtätigkeiten durch Automaten.\grqq\ \cite[Seite 7]{seidl_basiswissen_2012}
Das Spektrum der Testautomatisierung umfasst demnach alle Tätigkeiten die dazu dienen die Qualität einer Software zu Überprüfen. Darunter fallen alle Aufgaben die im Testprozess in den einzelnen Phasen des Softwareentwicklung anfallen.
Testautomatisierung ist also nicht nur auf die automatisierte Testdurchführung beschränkt, sondern kann ebenso bei der Testfallerstellung, der Testdatengenerierung, der Testauswertung oder auch der Testumgebungsherstellung und -wiederherstellung eine Rolle spielen.
\glqq Die Grenze der Automatisierung liegen [nach Seidl et al.] darin, dass diese nur die manuellen Tätigkeiten eines Testers übernehmen kann, nicht aber die intellektuelle, kreative und intuitive Dimension dieser Rolle.\grqq\ \cite[Seite 7]{seidl_basiswissen_2012}
Besonders lohnenswert ist eine Automatisierung bei sich wiederholenden Aufgaben, zu denen
besonders die Testdurchführung zählt. Die Automatisierung in diesem Bereich ist bereits weit verbreitet und soll auch den Fokus dieser Arbeit bilden. Die anderen Bereiche der Testautomatisierung sollen nur kurz aufgezeigt werden.
