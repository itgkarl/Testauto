\chapter{Einleitung}
\label{sec:einleitung}

Software hat in der heutigen Zeit eine hohe Verbreitung gefunden. Softwaresysteme werden immer wichtiger. Sowohl für Unternehmen als auch für jeden einzelnen persönlich. Die Anforderungen an moderne Software steigen ständig. Die Systeme werden immer größer und komplexer. Das hat zur Folge, dass auch die Anforderungen an die Qualität der Software immer weiter Steigen und wichtiger werden.
Fehler in Software verursachen immer wieder einen hohen finanziellen Schaden und können im schlimmsten Fall sogar Menschenleben kosten. Diese Probleme werden immer gravierender wenn mit der Wichtigkeit, Komplexität und Größe von Software nicht auch gleichzeitig die Qualität steigt. \cite{burnstein_practical_2003} \newline
Softwaretests sind ein weit verbreitetes Mittel um die Qualität einer Software zu überprüfen und sicherzustellen.
Komplexere und größere Software bedeutet daher auch gleichzeitig einen steigenden Testaufwand.\newline
Qualitätssicherungsmaßnahmen wie beispielsweise das Testen machen jetzt schon einen Großteil der Kosten in Softwareprojekten aus. Studien haben gezeigt, dass das Testen für 50\% und mehr der gesamten Projektkosten verantwortlich ist. \cite{ramler_economic_2006} 
Es ist daher nicht verwunderlich, dass es immer wieder versuche gibt diese Kosten zu reduzieren ohne dabei den angestrebten Qualitätsstandart zu reduzieren.
Einer der Wege die dafür vorgeschlagen wurden ist das automatisieren von Testfällen 
\cite{harrold_testing:_2000}
Im laufe der Jahre hat die Teatautomatisierung immer mehr an Bedeutung gewonnen. Heute ist sie bereits fester Bestandteil von Bereichen wie Continous Delivery und Continous Integration.

Diese Arbeit befasst sich daher mit der Automatisierung von Testfällen. Die Arbeit soll einen Überblick über die verschiedenen Bereiche der Testautomatisierung geben um dann den Bereich der Automatisierten GUI-Tests näher zu beleuchten.
Kapitel 1 ...





