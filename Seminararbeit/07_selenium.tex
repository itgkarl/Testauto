\chapter{Testautomatisierung mit Selenium}
\label{sec:testautomatisierung_mit_selenium}
Die Benutzeroberfläche ist als Schnittstelle für die Testautomatisierung weit verbreitet. Hierfür gibt es mehrere Gründe. Die Benutzeroberfläche ist sowohl für Tester, als auch für den Entwickler leicht greifbar und sehr anschaulich. Testfälle die über die Benutzeroberfläche arbeiten, kommen der realen Verwendung der Anwendung sehr nahe und die Dokumentation ist auf dieser Ebene meist am vollständigsten. Wird die Benutzeroberfläche für die Automatisierung verwendet, kommt das Vorgehen dem von klassischen Systemtests sehr nahe, da diese zumeist über die selbe Schnittstelle durchgeführt werden. \cite[vgl. Seite 48]{seidl_basiswissen_2012} Das fördert vor allem die Akzeptanz des Testautomatisierungsprojektes auf der Fachseite, da schnell sichtbare Erfolge erzielt werden können. 
Ein weit verbreitetes Tool für die Automatisierung von Tests gegen die Benutzeroberfläche ist Selenium.

\section{Selenium}
\label{sec:selenium}
Selenium \cite{selenium_selenium_????} ist eines der am weitesten verbreiteten Open-Source-Automatisierungswerkzeuge für Webapplikationen. Ordnet man das Tool gegen die in Kapitel \ref{sec:möglichkeiten_zur_testautomatisierung_in_erstellung_und_durchführung} beschriebene Unterteilung der Testautomatisierung ein, befindet sich Selenium in den unteren beiden Quadranten.
Selenium ist also ein Tool, das sowohl das manuelle Skripten von UI Tests, wie auch das halbautomatische \glqq recorde-and-playback\grqq\ unterstützt. Selenium ist genaugenommen aber kein einzelnes Tool. Der Begriff steht eher für eine Reihe von Komponenten mit unterschiedlicher Funktionalität, die der Testautomatisierung dienen. Dabei können folgende Teile unterschieden werden:

\begin{itemize}
	  \itemsep0pt
      \item \textbf{Selenium IDE} (Integrated Development Environment), eine Firefox-Extension, welche eine \glqq recorde-and-playback \grqq\ - Funktionalität beinhaltet. Mit der Selenium IDE ist es möglich Browserinteraktionen aufzuzeichen und die so erstellten Skripte zu editieren.
      \item \textbf{Selenium Core} enthält die komplette Basisfunktionalität von Selenium, also das Testbefehl-API und den Test-Runner. Mit Hilfe des Selenium Core können aufgenommene Skripte später wieder abgespielt werden.
      \item \textbf{Selenium WebDriver} ermöglicht es, Selenium aus Skripten und Programmiersprachen zu verwenden. Der WebDriver bietet dazu eine API, die es ermöglicht mit unterschiedlichen Browsern zu kommunizieren.
      \item \textbf{Selenium Grid} für die Parallelisierung von Testdurchläufen.     
\end{itemize}

Was Selenium also bietet, ist eine Reihe von Komponenten, die der Testautomatisierung in den Bereichen Testcodeerstellung \ref{subsec:testcodeerstellung} und Testdurchführung \ref{subsec:testdurchf\"uhrung} dienen.

\section{Testdesign mit Selenium}
\label{sec:Testdesign_mit_selenium}
Selenium bietet kein Werkzeug, dass den Tester in der Designphase des Testprozesse unterstützt.
Es ist mit Selenium also nicht möglich automatisiert Testfälle generieren zu lassen. Entscheidet man sich für Selenium als Testautomatisierungslösung, werden die Testfälle in der Regel wie in Kapitel \ref{subsec:testanalyse_und_design} Testanalyse und Design beschrieben, manuell erstellt. Selenium unterstützt zwar nicht den Desingprozess von Testfällen, versperrt jedoch auch nicht eine mögliche Automatisierung. Werden Testfälle, wie in Kapitel \ref{subsec:testdesign} beschrieben automatisch erstellt, könne sie durchaus später mit Hilfe von Selenium automatisiert werden. Beispielsweise wäre ein datengetriebener Automatisierungsansatz mit zuvor generierten Testeingaben durchaus denkbar. Das Zusammenspiel von beiden Phasen, also das automatische designen von Testfällen in Kombination mit einer automatischen Testfallgenerierung für die auch gleich automatisiert die Testskripte generiert werden, wie es beispielsweise modellbasierte Ansätze ermöglichen, ist jedoch ohne weiteres nicht möglich.

\section{Testcodeerstellung mit Selenium}
\label{sec:Testdesign}

Im Bereich von GUI-Applikationen stellen Webanwendungen einen Spezialfall dar. Anders als in den meisten Desktopanwendungen ist die Kernfunktionalität der Anwendung serverseitig realisiert. Als Kommunikationsschnittstelle zwischen Clients und Server dient HTML. Die eigentliche Benutzeroberfläche wird erst auf Clientseite durch den Browser aus dem gelieferten HTML-Dokument erzeugt. Das bietet für Testtools eine gute Basis, um auf die Elemente der Benutzeroberfläche zuzugreifen. \cite[vgl. Seite 59]{seidl_basiswissen_2012} UI-Testtools für Desktopanwendungen haben oft mit großen Herausforderungen zu kämpfen, um die einzelenen Grafikelemente einer Anwendung zu identifizieren. Selenium kann hier einfach auf Browserfunktionalitäten, bzw. auf die Struktur der HTML-Seite (Document Object Model) zurückgreifen. Über das Document Object Model können Tester die einzelnen Elemente einer Seite identifizieren und Interaktionen durchführen. Auf diese Weise kann der Workflow eines Testfalls in der Anwendung nachgebildet werden.

Selenium bietet dafür zwei unterschiedliche Möglichkeiten. Der Testcode zum Abbilden des Workflows kann halb automatisiert über eine \glqq recorde-and-playback\glqq\ - Funktionalität oder manuell erstellt werden.

\subsection{Recorde-and-playback}
\label{sec:recorde_and_playback}
Das automatisierte Aufnahmen von Testfällen ist eine der wichtigsten Funktionen der
Selenium IDE. Befindet sich die IDE im Aufnahmemodus, werden alle Interaktionen des Benutzers mit dem Browser gespeichert. Das so erstellte Testskript wird in einer Selenium eigenen Sprache gespeichert. Diese Sprache wird Selenese genannt. Die einzelnen Selenese-Kommandos werden in einer HTML-Tabellenstruktur abgelegt. Diese Tabellen mit Selenese-Kommandos können später wieder verwendet werden, um den gespeicherten Workflow erneut zu durchlaufen.
Selenium leidet jedoch unter den selben Problemen, unter denen die meisten \glqq recorde-and-playback\glqq\ -Tools leiden. Aufgenommene Testfälle sind in der Regel instabil.
Um die einzelnen Elemente der Website zu identifizieren, benutzt Selenium das Document Object Model der Website. Schon kleine Änderungen an der Oberfläche oder leicht veränderte Testdaten können die Struktur der Seite so stark verändern, dass die Testfälle die gespeicherten Elemente nicht mehr identifizieren können. Die über die \glqq recorde-and-playback \glqq -Funktionalität erstellte Testfälle erfordern daher eine vorgelagerte Planungs- und nachgelagerte Nachbearbeitungsphase, um die Testskripte auch für nachfolgende Durchläufe robust zu machen. Dieses Vorgehen widerspricht aber dem eigentlichen Grundgedanken von \glqq recorde-and-playback\glqq\ der darauf abzielt, die Testskripterstellung möglichst einfach und schnell zu gestalten.
\newline
Benutzt man für die Testskrtipterstellung die \glqq recorde-and-playback\glqq\ -Funktionalität der Selenium IDE, ist man nicht an die Sprace Selenese gebunden. Die IDE bietet die Funktionalität, die in Selenese erstellten Testskripte in eine Reihe von verschiedenen Programmiersprachen zu exportieren.
Die so erstellten Testskripte sind in ihrem Aufbau zu den Testskripten identisch die in Selenes generiert wurden und basieren immer auf einem xUnit-Framework der ausgewählten Sprache.
Sie leiden daher auch unter der selben Instabilität, unter der bereits die Selenese Testfälle leiden. Darüber hinaus ist der generierte Testcode wenig gekapselt und schwer wiederverwendbar. Das macht die generierten Skripte sehr Wartungsaufwändig und schlecht lesbar.
Diese Funktionalität sollte daher eher als Prototyping von Skripten verstanden werden.


\subsection{Manuell}
\label{sec:manuell}
Eine andere Möglichkeit ist das manuelle programmieren der Testskripte. Der manuell erstellte Testcode bedient sich den selben Tools, die auch die generierten und exportierten Testfälle aus der \glqq recorde-and-playback\glqq\ -Variante verwenden um mit der Anwendung zu kommunizieren. Das macht die Testfallerstellung zu Beginn, im Vergleich zur halbautomatischen Variante aufwendig. Dieser Mehraufwand kann allerdings durch deutlich bessere Wartbarkeit und Wiederverwendbarkeit schnell eingespart werden. Im Gegensatz du den generierten Testfällen kann, bei einem manuellen Ansatz von Anfang an, auf eine wartbare und wiederverwendbare Struktur in den Testfällen geachtet werden. Als best practice hat sich hierfür das Page Object Design Pattern \cite{selenium_test_????} durchgesetzt. Jede Maske der Anwendung wird als eigene Klasse umgesetzt. Diese Klassen beinhalten als einziger Ort in den Testskripten die Informationen über Funktionalität und Aufbau der Maske. So können redundanter Code und redundante Informationshaltung vermieden werden. Die Wartbarkeit und Robustheit der Testskripte steigt.


\section{Testdurchführung mit Selenium}
\label{sec:testdurchführung_mit_selenium}

Testfälle die mit Hilfe der \glqq recorde-and-playback\glqq\ -Funktionalität aufgenommen wurden, können direkt über die Selenium IDE oder auch in Selenium Core, einem HTML-basierten Durchführungswerkzeug automatisiert ausgeführt werden. 
Manuell erstellte bzw. exportierte Testfälle bedienen sich zur Ausführung dem entsprechenden xUnit Framework der benutzten Programmiersprache. Im Fall von Java wäre das beispielsweise JUnit oder TestNG. Selenium setzt für die Ausführung also auf gut verbreitete Frameworks, die im Bereich der Unit bzw. Integrationstests gegen die API-Schnittstelle bereits als Standartframeworks verwendet werden. Diese Frameworks sind den meisten Entwicklern bereits bekannt und gut in den Entwicklungsprozess integriert. In Java wird beispielsweise JUnit in allen gängigen IDEs durch Plugins unterstützt. Noch viel wichtiger ist jedoch, die bereits vorhandene Integration in den Buildprozess.
Hält man sich an Standarttools, wie Maven oder Gradle zum bauen seiner Projekte, können Testfälle die auf den gängigen xUnit-Frameworks basieren, direkt in den Buildprozess eingebunden werden.
Die automatisierte Ausführung der UI-Tests mit Selenium ist daher besonders einfach, da sie mit Standartmitteln möglich ist.

\section{Testabdeckung mit Selenium}
\label{sec:testabdeckung_mit_selenium}

Nach Abschluss der Tests stellt sich häufig die Frage: Wie gut sind meine Tests? Habe ich genug Testfälle, oder benötige ich noch weitere? Um diese Frage zu beantworten wird als Metrik häufig die Codeabdeckung durch die vorhandenen Testfälle herangezogen. Bei diesem Vorgehen wird gemessen, wie viel Prozent der Codezeilen der zu testenden Anwendung bei der Ausführung der vorhandenen Testfällen durchlaufen wird.
In herkömmlichen Testfällen, welche die API der Anwendung als Schnittstelle benutzen, ist das recht einfach möglich. Möchte man die Codeabdeckung seiner Unittests messen, können in Java beispielsweise Tools, wie EclEmma oder JaCoCo verwendet werden. Selenium verwendet als Schnittstelle allerdings nicht die API der Anwendung, sondern das UI. Das messen der Codeabdeckung ist in diesem Fall deutlich schwieriger. In der Regel wird gegen eine fertig gebaute Anwendung getestet. Es muss also darauf geachtet werden, dass der durchlaufene Code in der Anwendung und nicht im Testprojekt gemessen wird.
Das erweist sich meist als deutlich schwierigere Aufgabe, als das Messen beim Verwenden der API. Das Vorgehen, um eine Messung der Abdeckung in einem Existierenden Projekt zu erreichen, ist stark von der Programmiersprache, in der die zu testende Anwendung entwickelt wurde und von den verwendeten Testtools abhängig. Im Java ist hierfür beispielsweise eine Bytcode Manipulation der zu testenden Anwendung notwendig, bei der jeder Klasse, in der die Codeabdeckung gemessen werden soll, die entsprechenden Anweisungen injiziert werden.
Das messen der Codeabdeckung in ausgelagerten Projekten ist also prinzipiell möglich und wird auch von Tools wie EclEmma unterstützt, erweist sich aber als deutlich schwieriger als beim Testen gegen die API.


\chapter{Ausblick}
\label{sec:ausblick}

Das Thema Testautomatisierung wird auch in den kommenden Jahren ein wichtiges Thema im Bereich der Softwareentwicklung bleiben. Vor allem Continous Integration und Continous Delivery werden hier weiter eine treibende Kraft sein. Diese zwei Bereiche sind als Entwicklungskonzepte derzeit sehr stark im Trend, was auch der Verlauf der Suchanfragen, in den letzten Jahren, zeigt.
\begin{figure}[htb]
  \centering  
  \includegraphics[scale=0.75]{img/cdTrend.JPG}\\
  \footnotesize\sffamily\textbf{Quelle:} \cite{google_google_2014}
  \caption{Trend der Suchanfragen nach dem Begriffen \glqq continous integration\grqq\ und \glqq continous delivery\grqq\ seit 2006}
  \label{fig:cdTrendTestauto}
\end{figure}
Die Testautomatisierung ist ein integraler Bestandteil dieser Entwicklungskonzepte und wird daher von diesem Trend mitgetragen. Unittests bzw. Integrationstests mit den gängigen XUnit-Frameworks, haben sich bereits in vielen Projekten durchgesetzt. Für eine sinvolle Continous Integration bzw. Continous Delivery müssen jedoch alle Bereiche, von Unit- bis Systemtest, durch eine Automatisierung abgedeckt werden. Es ist daher zu erwarten, dass auch die Automatisierung in diesen Bereichen für viele Projekte an Interesse gewinnen wird.


Im Bereich der Automatisierungstools wird Selenium auch in der nächsten Zeit eine weit verbreitete Lösung darstellen. Der Verlauf der Suchanfragen in den letzten Jahren zeigt, dass Selenium im Vergleich zu anderen gängigen Lösungen immer noch sehr beliebt ist.

\begin{figure}[htb]
  \centering  
  \includegraphics[scale=0.75]{img/autoToolsTrend.JPG}\\
  \footnotesize\sffamily\textbf{Quelle:} \cite{google_google_2014}
  \caption{Trend der Suchanfragen nach gängigen Automatisierungstools}
  \label{fig:cdTrendSelenium}
\end{figure}